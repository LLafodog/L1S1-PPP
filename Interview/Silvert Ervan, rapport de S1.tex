\documentclass[12pt, a4paper,draft]{report} %ne pas oublier de retirer draft

\usepackage[utf8]{inputenc}
\usepackage[T1]{fontenc}
\usepackage[francais]{babel}

% ====
\usepackage{hyperref}
\usepackage{shorttoc}

\author{Ervan Silvert}
\title{Dossier Projet Professionnel \\ L1 Semestre 1}
\date{}

\begin{document}


% ============ VARIABLES ================

% ============ COMMANDS ===============
\newcommand{\finalword}[1]{\\ \begin{center} \textsc{#1} \end{center} }





\maketitle

\shorttoc{Sommaire}{5}

% ================INTRODUCTION =================
\chapter{Introduction}

\section*{Explication du sujet}
Bienvenue dans mon dossier professionnel de premier semestre servant à l'introduction au monde de l'entreprise. Le but est d'entre-apercevoir le monde du travail par le biais d'une interview d'un professionnel pratiquant un travail susceptible de nous plaire et de diverses études de son implémentation économique.

Pour cela, nous avons été amenés à nous étudier, comprendre nos motivations pour savoir plus précisément où nous aimerions finir après nos études. La prochaine section sera donc dédiée à mon introspection.

\section{Introspection}
Partons de la base: pourquoi suis-je allé en Informatique ? La réponse me vient aisément.

En 2012 un ami m'a montré ce qu'il faisait de son temps libre. Il ouvrait un logiciel et tapait des lignes de codes sans que je ne comprenne rien. \`A la fin, un mini-jeu apparaissait et j'avais des étoiles plein les yeux. Depuis, je suis pris de passion pour le code, j'ai appris le C++ en autodidacte, ai appris avec une librairie graphique et me consacre depuis peu à un gros projet de RPG\footnote{Role Play Game, un style de jeu consistant à incarner un personnage dans une histoire donnée.}.

Dès lors, après une erreur d'orientation me voici en informatique, prêt à compléter mon savoir et réorganiser tout ce que j'ai pu apprendre, apprendre à apprendre et m'organiser. Seulement j'ai peur de ne pas pouvoir accèder au domaine du jeu vidéo. Il en découle un besoin de trouver un autre centre d'intérêt, une voie de secours en cas d'échec et c'est ici que le PPP\footnote{Projet Personnel et Professionnel, une unité de mon cursus} m'a aidé. Voici concrètement les metiers qui semblent me plaire, dans l'ordre décroissant d'intérêt.

\paragraph{Programmeur logiciel}
Concrètement, c'est de la conception et du code, tout ce que j'aime. Toutefois, j'ai peur de tomber dans la situation déplaisante que serait celle de rester plusieurs mois sur un logiciel qui m'ennuie. Reste donc à bien choisir le poste. \`A vrai dire, je pense que le plaisir que j'éprouve à voir un projet grandir et vivre, devenir fonctionnel et cohérent l'emporterait sur l'ennui du logiciel lui-même. Peu m'importe d'être sous des ordres, j'aime voir le résultat croître et comprendre pourquoi tout cela est possible. 

Mon premier programme \small{\emph{(qu'il repose en paix)}} m'a épaté; comment une centaine d'intrusctions a-t-elle pu donner vie à un bonhomme se déplaçant sur des briques ? C'était un genre de Mario mais j'en ai rêvé pendant longtemps, depuis ce jeu je veux toujours faire plus et je pense que ce sentiment pourrait être un point très positif dans ce travail.

\paragraph{Webmaster}
Ce qui m'a poussé à écrire ce travail dans cette liste est arrivé vite. En C.M.I. \footnote{Cursus Master en Ingénièrie, le cursus d'excellence que je suis.} Programmation nous avons dû créer un site personnel avec comme thème imposé le \og C.V. Numérique à la fin de nos études \fg. Je pensais que le langage du web, par sa caractéristique plus descriptive que fonctionnelle, me déplairait; hors en un après-midi j'ai réalisé un site qui était cohérent, beau et j'ai pu créer exactement ce que je souhaitais. Encore une fois cette fierté de donner naissance m'émut et je dois avouer qu'en faire un métier peut vraiment être passionnant. D'autant plus en utilisant le javascript qui semble être la seule chose qui manque à ma connaissance actuel du HTML/CSS pour que je l'apprécie grandement.

Récemment, j'ai participé à \og La Nuit de l'Info 2014 \fg . C'était la première fois que je programmais dans une équipe, qu'il y avait une vraie organisation. Nous reparlerons plus en détail de tout cela plus tard, mais dans ce concours j'ai été ammené à programmer le site internet supportant le programme majeur conçu par les Master; ca a été pour moi un vrai plaisir. En faire mon métier pourrait vraiment être intéressant !

D'ailleurs, j'ai pu m'apercevoir au long de mes longues promenades sur la toile que finalement le webdisgn était très recherché par les professionnels. J'ai souvent vu des offres de récompenses pour des sites entièrement préparés pour de nouvelles entreprises; je pense que le marché sur ce point est très ouvert aux débutants. 

\paragraph{Sécurité Web}
Ce métier en revanche est une nouveauté dans mes goûts, je ne sais vraiment pas si cela me plairait effectivement. Dans les actes, je me rend compte qu'à mesure des programmer des bugs \og invisibles \fg semblent apparaître~\emph{(souvent dus à des négligences comme des > à la place de >= )}, hors les tracker est devenu amusant, très plaisant. De fait, si on considère un réseau internet comme un programme et ses fuites commes les bugs, je pense que ce travail peut être des plus agréables pour moi.

\section{Bilan}
De très loin, mon passe-temps devenu passion étant la programmation d'un jeu vidéo de son départ à la fin, il est évident que ce rapport portera sur ce métier; à savoir: 
\finalword{ Programmeur jeu vidéo !} 

% ================ Étude  intrinsèque du métier =================
\chapter{Étude intrinsèque du métier }
\section*{Remarques Préalables}
Tout au long de ce chapitre nous nous appuirons sur les interview de plusieurs professionnels du métier, en se concentrant avant tout sur l'interview principales de M. Antoine Guerchais. Une section finale sera dédiées au coordonnées de ces généreux volontaires qui m'ont accordé de leur temps.

\section{Démarches de recherches}
Le domaine du jeu vidéo étant peu développé dans la région de Franche-comté, j'ai cherché un site servant de hub aux professionnels du domaine. Après quelques recherches, le site de l' \href{http://www.afjv.com/index.php}{AFJV}\footnote{Association Français du Jeu Vidéo} s'est imposé.


\section{Présentation des professionnels}
\subsection[Antoine Guerchais]{Antoine Guerchais, le généreux.}
Je vous présente la personne la plus agréable que je n'ai jamais rencontré.

\section{Point de vue technique}
\subsection{Compétences Nécessaires}
\subsection{Conditions de travail}

\section{Point de vue fonctionnel}
\subsection{Étude d'une Journée type}
\subsection{Étude du schéma type}
\subsection{Perspectives possibles d'évolution}

\section{Point de vue pécunier}
\section{Coordonnées}


%coordonnée

% ================ Étude marchétaire =================
\chapter{Étude marchétaire} 
\section{Profil type du candidat}
\section{Profil géographique}
\section{Évolution à attendre} %notez la majuscule accentuée
\section{Exemples d'offres}

% ================ Comparaison avec mon propre profil=================
\section{Situation actuelle}
%Opposition à:
\section{Situation escomptée}

\section{Démarches entreprises}
%oppositions à:
\section{Démarches à entreprendre}

%Inclure des exemples court,moyen,long terme



%%Conclure sur le fait de monter son entreprise ?

\begin{abstract} %résumé
\end{abstract}

\end{document}
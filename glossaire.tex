\newacronym{ppp}{PPP}{Projet Personnel et Profesionnel}
\newacronym{afjv}{AFJV}{Association Français du Jeu Vidéo}
\newacronym{poo}{POO}{Programmation Orientée Objet}
\newacronym{ide}{IDE}{Integrated Development Environment ou Environnement de développement}
\newacronym{esa}{ESA}{Entertainment Software Association}

\newglossaryentry{POO}
                 {
                   name = {Programmation Orientée Objet},
                   description= {Méthode de programmation majoritairement adopté par les développeurs logiciels.}
                 }

\newglossaryentry{SCRUM}
{
name = {SCRUM},
description = {Scrum est considéré comme une méthode agile.\\ La méthode s'appuie sur le découpage d'un projet en boîtes de temps, nommés « sprints ». Les sprints peuvent durer entre quelques heures et un mois (avec une préférence pour deux semaines). Chaque sprint commence par une estimation suivie d'une planification opérationnelle. Le sprint se termine par une démonstration de ce qui a été achevé et contribue à augmenter la valeur d'affaires du produit. Avant de démarrer un nouveau sprint, l'équipe réalise une rétrospective : elle analyse ce qui s'est passé durant ce sprint, afin de s'améliorer pour le prochain. L'adaptation et la réactivité de l'équipe de développement est facilitée par son auto-organisation. - Source: wikipédia }
}

\newglossaryentry{Cenv}
{
name = {Cycle en V},
description = {
Le modèle du cycle en V (en comparaison avec les méthodes dites Agile) est un modèle conceptuel de gestion de projet imaginé suite au problème de réactivité du modèle en cascade. Il permet, en cas d'anomalie, de limiter un retour aux étapes précédentes. \\
Les étapes : \\
    Analyse des besoins et faisabilité \\
    Spécification fonctionnelle \\
    Conception architecturale\\
    Conception détaillée\\
    Codage\\
    Test unitaire \\
    Test d'intégration\\
    Test de validation : recette usine, validation usine, VAU\\
    Test d'Acceptation : vérification d'aptitude au bon fonctionnement, VABF\\}
}

